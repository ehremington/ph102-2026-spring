\week \ covers sections of chapter 18 in the textbook. Topics include:

\begin{itemize}
	\item resistive circuits and Ohm's Law
	\item power use in circuits
	\item networks of resistors
\end{itemize}

\begin{enumerate}
	\setlength\itemsep{2 in}
	
	\item
	Charges are flowing through a wire like water through a pipe. If the wire has \SI{5}{\ampere} of current flowing through it, then how much charge passes a point in the wire in \SI{10}{s}?
	
	\item
	Batteries are often have a rating written on them. A Energizer AA battery has a rating of \SI{3000}{milliAmp \cdot hours} (\si{\milli\ampere\hour}). 
	\begin{enumerate}
	\setlength\itemsep{1.25 in}
	\item What kind of quantity is a \si{milliAmp \cdot hour}? What base unit is this related to? 
	\item If the battery discharged all of this charge at \SI{1.5}{\volt}, then how much potential energy was initially in the battery? 
	\item If the battery discharges at a rate of \SI{25}{\milli\ampere}, then how long will the battery last? 
	\item What is the power output of the battery? 
	\end{enumerate}

   \item A simple particle accelerator can be designed from a capacitor in a vacuum. The negatively charged plate has a heater near it, and when electrons have enough thermal energy, then they are released from the plate and are accelerated in the electric field between the plates. If the positive plate has a hole in it, then the electrons will fly through and now you have an electron gun. If the potential on the plates is maintained, then a steady stream of electrons can result. If a mole of electrons passes through the hole in the positively charged plate per second, then what is the current of the gun? Bonus question from last week, if voltage between the plates is \SI{5000}{\volt} then how fast will the electrons emerge from the gun?
   
   \item
    The resistance of a wire is directly proportional to the length and inversely proportional to the cross-sectional area. But measuring the area of a wire really comes from measuring the diameter of a wire and then using that to find the area. So use the area of a circle formula to substitute into the resistance of a wire equation and make sure you express this in terms of diameter. If the diameter of one wire is twice the diameter of another, then by what factor are the resistances related?
    
    \item
    A copper wire is \SI{100}{m} long and has a diameter of \SI{1}{mm}. It is connected to a \SI{12}{\volt} battery. 
    \begin{enumerate}
    	\setlength\itemsep{1.25 in}
    	\item What is the resistance of the wire?
    	\item How much current goes through the wire?
    	\item What is the power supplied by the battery?
        \item What power is dissipated by the wire?
        \item How much heat would this give off in 10 minutes?
        \item How much would this heat up a liter of water?
        \item If the length of the wire doubled, what would happen to all of the above quantities?
    \end{enumerate}

	\item
	\SI{100}{\meter} of a certain wire has a resistance of \SI{5}{\ohm}. What is the resistance of the same wire if it is \SI{12}{\meter} long?
	
	\item 
	If you stretch a wire to three times its original length, then the wire will get thinner, since it is the volume that remains the same. So by what factor does the cross sectional area change (assume as with all wires that it is a cylinder)? By what factor does the resistance change?
	
	\item
	Combine the equations for the power dissipated by a resistor, $P = I V$ and Ohm's Law $V = I R$, in two ways and get two equations for power out. One should express power in terms of current and resistance, and the other should express power in terms of voltage and resistance. 
	
	\item
	You have a \SI{12}{\volt} battery connected to a \SI{120}{\ohm} resistor. Draw a picture of this. If you measure the voltage from the negative terminal to the positive terminal of the battery, what would you read? If you measured from the positive to the negative, what would you read? If you measured from the positive terminal of the battery to the side of the resistor near to it, what would you measure? If you measure across the resistor from the side nearer to the positive terminal to the other side, what would you measure? How much current is going through the resistor? What power is dissipated by the resistor? \newpage
	
	\item 
	Now let's make two changes to the previous problem. Instead of a \SI{12}{\volt} battery you have a \SI{9}{V} and two \SI{1.5}{\volt} batteries \emph{in series}. Then instead of a single resistor, there are 3: one that is \SI{20}{\ohm}, one \SI{40}{\ohm} and one \SI{60}{\ohm} all in series. Draw a picture of this first. What is the equivalent resistance of this set of resistors? How much current flows through each one of these resistors?  What is the voltage drop across the entire group of resistors? What is the voltage drop across each one? How much power is dissipated by the entire group of resistors? How much power is dissipated by each one? \newpage
	
	\item
    Now lets take the same \SI{20}{\ohm}, \SI{40}{\ohm} and \SI{60}{\ohm} resistors and connect them each \emph{in parallel}, that is they each have their own connection to the \SI{12}{\volt} battery. Draw this out as a circuit diagram. What is the potential drop across each resistor? What is the current through each resistor? What is the total current supplied by the battery? What is the equivalent resistance of this circuit? How much power is supplied by the battery? How much power is dissipated by each resistor?
    \newpage 
	
	\item
	For the resistor network below, what is the equivalent resistance? What is the current through $R_1$? What is the voltage drop across $R_1$? What is the voltage drop across $R_2$ and $R_3$? What is the current through $R_2$ and $R_3$? 
	
	\includegraphics[scale=.5]{figures/resistors-combo.png}
		
	\item
	A light bulb is rated as \SI{75}{\watt}. If it is connected to a standard \SI{120}{\volt} then what current does it draw from the outlet? What is the resistance of the filament in the bulb?
	
   \item
   Household wiring is done so that each outlet is in parallel to the others. A group of several outlets are all connected to a circuit breaker, which will typically break the circuit when more than \SI{20}{\ampere} flow down the wire to that group of outlets. The voltage provided to the circuit is \SI{120}{volts} (although unlike the constant voltage provided by a battery, the voltage out of the wall is alternating and looks like a sine wave, but this is not important for us in this problem). Most appliances are rated by the power that they do work, so for example a vacuum cleaner may operate at \SI{700}{\watt} and a coffee maker at \SI{900}{\watt} and a toaster might be \SI{1000}{\watt}. If these are all plugged in to outlets on the same circuit, how much current would be supplied through the breaker? Will it trip and stop the current?
   
   \item
   When you pay your power bill, you pay for each \si{kiloWatt\cdot hour} of energy that you use. How much energy in \si{Joules} is one \si{\kilo \watt\cdot \hour}?
	
	
	
	
	\item Assume each atom of copper in a wire contributes one electron to be able to move freely in a current. The mass density of copper is $\rho_m = \SI{8.92}{g/cm^3}$ and the atomic mass of copper is \SI{63.5}{g/mole}. What is the number density of conduction electrons in copper (electrons per m$^3$)? If \SI{5}{\ampere} of current are flowing through a copper wire that has a diameter of \SI{0.1}{mm}, then what is the drift velocity of electrons in the wire?



\newpage 

\ % The empty page

\newpage

\end{enumerate}