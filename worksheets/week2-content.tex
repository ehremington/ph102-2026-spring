Week 2 covers sections of chapter 14 in the textbook. Topics include:

\begin{itemize}
	\item heat, energy, and power
	\item heat capacity, specific heat capacity, and molar heat capacity
	\item latent heat of fusion or vaporization
\end{itemize}


\begin{enumerate}
\setlength\itemsep{2 in}

\item
Suppose you have \SI{2}{\kilogram} of water in a cup and you put it in the microwave. What form of heating is this (conduction, convection, or radiation)? It takes 1 minute to heat up from \SI{20}{\celsius} to \SI{40}{\celsius}. What is the temperature change in Celsius ($\Delta T$)? What is the change in temperature in Kelvin? Look up the specific heat of water and be careful to specify the units. How much heat would it take to accomplish this temperature change?

\item
What are the units of power? What is another way to express those units? To follow up the previous problem, what is the power delivered to the water by the microwave? If this rate of heat delivery continues, how long will it take for the water to reach its boiling point?

\item
You have three samples of material, all 1 kg each. The materials are gold, copper, and aluminum. Put these in order of how much heat is necessary to change the temperature 1 degree. 

\item
You have the same materials as in the last problem. If you put the same amount of heat into each sample, put them in order of which would heat up the most (remember that when we say "heat up" we mean increase in temperature).

\item
How much heat does it take to bring \SI{2}{\kilogram} of aluminum from \SI{25}{\celsius} to \SI{50}{\celsius}?

\item
If it takes \SI{5000}{\joule} to bring an ingot of gold from \SI{25}{\celsius} to \SI{50}{\celsius}, then what mass of gold is the ingot?

\item
If \SI{5000}{\joule} of heat goes into a \SI{1.5}{\kilogram} glass dish that was initially at \SI{20}{\celsius}, then what is its final temperature?

\item
If a \SI{5}{kg} block of aluminum has a temperature of \SI{500}{\celsius}, how much heat does it give off to cool down to \SI{490}{\celsius}? What should the sign of heat be in this case? 

\item If the aluminum block in the above problem gave up this heat because it was put in contact with a cooler \SI{5}{\kilogram} block of lead, then by how much does the temperature of the lead rise? If the lead was originally \SI{75}{\celsius}, then what would be its new temperature?

\item
Keep this process going, the aluminum cooling off by \SI{10}{degress} and the lead warming up by whatever you found in the above problem. Approximately what temperature would they meet? This temperature is known as the \emph{equilibrium temperature} since after this heat flow stops, and they remain at the same temperature (ignoring heat that they lose to the surroundings).

\item
Lets do this in one step now. If the aluminum has an initial temperature of \SI{500}{\celsius} and an \emph{unknown final temperature}, and the lead starts at \SI{75}{\celsius} and has an unknown final temperature but the same final temperature as the aluminum, since that is the equilibrium temperature, then how can we find this with one expression? (\emph{Hint: the overall energy of the system does not change. So any change in energy of one plus the change in energy of the other must be zero.})

\item
The exact same logic that applied to the above problem, applies to mixing two substances together. You can still treat them as separate substances with one or more giving up energy in the form of heat to the others. It is always assumed that the materials are kept in a well insulated container so that no heat is lost to the surroundings. This is a good way of measuring the heat capacity of an unknown material by mixing it with a material of known initial temperature and heat capacity and then measuring the equilibrium temperature that results. So suppose you start with \SI{100}{\gram} of water at \SI{90}{\celsius}, and you pour in \SI{50}{\gram} of unknown metal at \SI{20}{\celsius}. You stir the mixture and notice that the temperature of the mixture comes down to \SI{40}{\celsius} and then remains at that temperature. What is the specific heat of the unknown material?

\item
A monatomic gas is in a container that keeps the gas at constant volume of \SI{0.01}{m^3}. The gas is at room temperature and the pressure is \SI{1}{atm}. How much heat do you have to add to increase the temperature by \SI{10}{\celsius}? What is the pressure in the container at that temperature?

\item
A monatomic gas is in a container that keeps the gas at constant pressure of \SI{1}{atm}. The gas is at room temperature and the volume is \SI{0.01}{m^3}. How much heat do you have to add to increase the temperature by \SI{10}{\celsius}? What is the volume of the container at that temperature? 

\item
Why is there a difference between the heat needed to change the temperature of an ideal gas at constant volume vs at constant pressure?

\item
How much heat is needed to melt a block of ice that is \SI{10}{kg} at \SI{0}{\celsius}? What if the ice starts at \SI{-10}{\celsius}?

\item
You drop a \SI{0.1}{kg} ice cube at \SI{0}{\celsius} into a cup with \SI{200}{\gram} water at room temperature (\SI{20}{\celsius}). Before you work this problem, think about the possible outcomes. How much heat does it take to melt the ice cube completely? How much heat would it take to cool the water to \SI{0}{\celsius}? Given this information which outcome is the one that happens? Find the equilibrium temperature.  

\item
A \SI{60}{\kilogram} hiker wished to climb to the summit of Mt. Ogden, an ascent of \SI{5000}{vertical feet} (\SI{1500}{\meter}). 
\begin{itemize}
\setlength\itemsep{2 in}
\item How much work will it take for her to reach this height?
\item Assuming that she is only 25\% efficient at converting chemical energy from food into mechanical work, and that essentially all of the mechanical work is used to climb vertically, roughly how many bowls of corn flakes should the hiker eat before setting out? (standard serving size \SI{1}{oz}, \SI{100}{Calories})
\item As the hiker climbs the mountain the other 75\% of the energy from the corn flakes is converted into thermal energy. If there were no way to dissipate this energy, by how many degrees would her body temperature increase? (Assume the human body is mostly water so that it has the same specific heat as water.)
\item In fact, the extra energy does not warm the hiker's body significantly; instead, it goes (mostly) into evaporating water from her skin and within her lungs. How many liters of water should she drink during the hike to replace the lost fluids? (At \SI{25}{\celsius}, a reasonable outdoor temperature to assume, the latent heat of vaporization of water is \SI{580}{cal/g}, 8\% more that at \SI{100}{\celsius}.)
\end{itemize}

\newpage 

\ % The empty page

\newpage

\end{enumerate}