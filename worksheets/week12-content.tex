\week \ covers sections of chapter 23 in the textbook. Topics include:

\begin{itemize}
	\item law of reflection
	\item reflection from plane mirrors
	\item reflection from convex and concave mirrors
	\item image formation
	\item ray tracing

\end{itemize}

\begin{enumerate}
\setlength\itemsep{2 in}

\item 
When a plane mirror is hung on a wall, then what conditions must be met about it size and placement so that you can see your entire body in the mirror? Draw the path that light takes from your feet to your eyes. Draw in the path that light takes from the top of your head to your eyes. 

\includegraphics[scale=.45]{figures/mirrorHeight.png}

\item
Two mirrors are perpendicular to each other and meet at a corner, one horizontal and one vertical. Light is incident at the horizontal mirror at an angle of 30 degrees from the normal. Draw this picture and then find at what angle the light leaves the other mirror?

\item
The focal point of a convex mirror is the point midway between the center of the circle that defines the mirror and the vertex of the mirror or where it intersects the principle optical axis. Draw this out and identify the focal point. Now place an object somewhere along the optical axis. Now let's do a ray tracing to find the image. Only two rays are needed but the third can confirm.
\begin{enumerate}
	\item Draw a ray parallel to the principle axis from the top of your object to the mirror. This will reflect as if it came from the focal point.
	\item Draw a ray from the top of your object toward the center of the circle to the mirror. This ray will reflect on itself as if it came from the center of the circle.
	\item (Optional) Draw a toward the focal point. It will reflect parallel to the principle axis.
\end{enumerate}
Find the point where all of these lines appear to intersect. Is your image upright or inverted? Would its image distance be positive or negative? Is this a real or virtual image? Is the focal length be positive or negative?\giantskip

\item
Now let's work a similar example with the mirror and magnification equations. Draw out a convex mirror with a radius of curvature of \SI{4}{cm} and put an object \SI{10}{cm} away from the mirror. The object has a height of \SI{2}{cm}. Use both a ray tracing and the equations to find the image distance, the image height?

\item
For a convex mirror with a focal length of \SI{-2}{cm}, plot image distance as a function of object distance. Will there ever be a real image according to your graph? What value does the image distance approach? \bigskip

\item
Now let's work with a concave mirror. Draw a circle and identify its center, the mirrored surface, the principle axis, and the focal point. Draw an object on the principle axis farther away from the mirror than the center of the circle. Now let's do a ray tracing to find the location and size of this image.
\begin{enumerate}
	\item Draw a ray of light from the top of the object to the mirror parallel to the principle axis. This ray will reflect through the focal point.
	\item Draw a ray of light from the top of the object through the focal point to the mirror. This ray will reflect parallel to the optical axis.
	\item (Optional) Draw a ray through the center of the circle to the mirror. This radial ray will reflect on itself backwards. 
\end{enumerate}
Find the point where these rays intersect. This is the location of the image. Draw the image from the principle axis to the intersection point. Is the image upright or inverted? Would this image distance be positive or negative?  Is this a real or virtual image? Is the focal length be positive or negative?\giantskip

\item
Now work out the above example using the mirror equation and magnification equation. For a concave mirror that has a radius of curvature of \SI{4}{cm}, what is the focal length? If the object is \SI{10}{cm} away from the vertex of the mirror and \SI{2}{cm} in height, then where is the image formed and what is it height? \bigskip

\item
 Draw a plot of the image distance as a function of object distance for a concave mirror that has a focal length of \SI{2}{cm}. What can you say about the image distance for object distances larger than the radius of the mirror? What about image distances for object distances less than the radius of the mirror but more than the focal point? What happens to the image distance for object distances less than the focal length? What does the image distance approach for very large object distances?

\newpage 

\ % The empty page
\ruler{rightup}{15cm}

\newpage

\end{enumerate}