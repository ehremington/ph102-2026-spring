\week \ covers sections of chapter 17 in the textbook. Topics include:

\begin{itemize}
	\item work and electric potential energy
	\item electric potential or voltage
    \item conservation of energy
    \item electric field lines and equipotential surfaces
	\item energy storage in capacitors
\end{itemize}

\begin{enumerate}
\setlength\itemsep{2 in}
	
\item An uniform electric field has a strength of \SI{1000}{N/C} and points in the positive x-direction. It doesn't really matter what is causing the electric field but you can imagine we are inside of a capacitor. A \SI{1}{\micro\coulomb} charge is moved around within this field by an external force. 
\begin{enumerate}
\setlength\itemsep{1 in}
	\item How much work does it take to move the charge from the position $x=\SI{5}{cm}$ to a position of $x=\SI{1}{cm}$?
	\item How much work does it take to move the charge from the position $x=\SI{5}{cm}$ to a position of $x=\SI{9}{cm}$ 
	\item How much work does it take to move the charge from the position $x=\SI{-5}{cm}$ to a position of $x=\SI{1}{cm}$
	\item How much work does it take to move the charge from the position $y=\SI{0}{cm}$ to a position of $y=\SI{5}{cm}$
\end{enumerate}

\item 
For the above problem, what is the amount of work \emph{per unit of charge} that was done in each step?\bigskip

\item
A fixed point charge of \SI{10}{\micro\coulomb} creates an electric field throughout space around it. 
\begin{enumerate}
   \setlength\itemsep{1 in}

   \item How much energy would it take to move a \SI{1}{\micro C} charge from infinitely far away to a position of \SI{100}{\meter} away?
   \item How much energy would it take to move a \SI{1}{\micro C} charge from infinitely far away to a position of \SI{90}{\meter} away?
   \item How much energy would it take to move a \SI{1}{\micro C} charge from a position \SI{100}{\meter} away to a position of \SI{90}{\meter} away?
   \item To move \SI{10}{\meter} closer, how much more energy would it take?
   \item How much energy would it take for each \SI{10}{\meter} displacement closer to the fixed charge?
\end{enumerate}

\item
Work the steps in the above problem for the voltage change rather than the energy change. Also plot a graph of voltage vs. radius and illustrate on the graph what it looks like to move from say \SI{50}{\meter} to \SI{40}{\meter} in terms of the voltage difference.\newpage

\item 
Now imagine that the source charge is negative. What does this change about all of the answers to the previous problem?\giantskip

\item
What are some inside out versions of the previous problems. (Write out the wording and I'll feature some of them in the space below.)

\item
In order to find the voltage at a place around multiple charges, you simply find the voltage at a location due to each charge individually and then add them up. Same goes for finding the potential energy of putting a charge at a place around other charges. So in the figure below we have $q_1 = \SI{+20}{\micro\coulomb}$ and $q_2 = \SI{-5}{\micro\coulomb}$ each \SI{1}{\meter} away from the origin. \newline
\includegraphics[scale=0.7]{figures/charges-on-xaxis-potential.png}
\begin{enumerate}
\setlength\itemsep{1 in}
\item What is the voltage at the location of $x=\SI{+2}{\meter}$ on the x-axis relative to infinitely far away?
\item What is the voltage at the location of $x=\SI{-2}{\meter}$ on the x-axis relative to infinitely far away?
\item What is the voltage at the location of $x=\SI{-2}{\meter}$ on the x-axis relative to the location of $x=\SI{+2}{\meter}$ on the x-axis?
\item What is the voltage at the location of $x=\SI{+1}{\meter}, y=\SI{+1}{\meter}$ (again relative to infinitely far away; if I ever forget to say this then this is what I mean)?
\item Where along the x-axis would the voltage be equal to zero? What does this mean for the amount of work it would take to move a charge to that location from infinitely far away?
\end{enumerate}

\item
Now we need to put together the change in potential energy with the kinetic energy so that we can use the conservation of energy. To review some equations that we used last semester, lets start with a statement of the conservation of energy 
\begin{align*}
	 K_f + U_f = K_i + U_i + W_{\textit{non-conservative}}
\end{align*} 
The only non-conservative forces we will have are the ones from outside that increase the potential energy of a charge, but there will be no friction.  So if a charge starts from rest at some place you move it to another place and it arrives there at rest then the equation becomes:
\begin{align*}
	0 + U_f &= 0 + U_i +W_{\textit{nc}} \\
	\Delta U &= W_{\textit{nc}}
\end{align*}
which is what we have been using this entire time.

But, if instead of doing work to move it, what if we just let go? The charge would begin to move in response to the electric field that it was in and the potential energy would go down, and in its place the kinetic energy would increase since the charge is speeding up. There are several ways to represent this:
\begin{align*}
	K_f + U_f &= K_i + U_i + 0\\
	\Delta K &= -\Delta U
\end{align*}
Ok now for an actual question. Take this last expression and do a substitution to express the change in kinetic energy as a change in voltage.

\item
A \SI{-10}{\micro\coulomb} charge is in a uniform electric field of \SI{1000}{\newton/\coulomb} that points to the right. If the charge travels a distance of 1 cm \emph{to the left}, then how much has the voltage changed within the field? How much has the potential energy changed? How much has the kinetic energy changed? The charge has a mass of \SI{1}{\milli\gram}. How fast is the charge going after it has traveled this distance? What would be different if the charge moved to the right?

\item
Draw two plates of opposite charge (a capacitor) and an electric field within it pointing to the right. What do the equipotential surfaces look like within this capacitor? If the electric field within the capacitor is \SI{500}{N/C} and the capacitor plates are \SI{1}{cm} apart, then what is the voltage between the two plates? Divide this voltage difference into a sensible number of ``steps'' of voltage. Make these the equipotential surfaces and then find the distance between surfaces. Add all of this to your drawing.\hugeskip

\item
Refer to your drawing in the above problem. If you put a positive charge in between the plates of this capacitor which way will it accelerate? Is this in the direction of increasing or decreasing voltage or neither? What about a negative charge? The equation for relating the potential energy and the voltage does need a sign on the charge.

\item 
One more point charge question. Suppose a \SI{+20}{\micro\coulomb} point charge is fixed in place, and another charge of \SI{+1}{\micro\coulomb} is fired directly at it with a velocity of \SI{100}{\meter\per\second} from a distance of \SI{10}{\meter} away. First, what is the voltage at this position of \SI{10}{\meter} away? Then, where does this charge come to a stop? Then, what happens? How fast is it going when it is \SI{20}{\meter} away? What about when it is ``infinity'' meters away?

%\item
%IN PROGRESS...


\newpage 

\ % The empty page

\newpage

\end{enumerate}